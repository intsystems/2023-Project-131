\documentclass{beamer}
\beamertemplatenavigationsymbolsempty
\usecolortheme{beaver}
\setbeamertemplate{blocks}[rounded=true, shadow=true]
\setbeamertemplate{footline}[page number]
%
\usepackage[utf8]{inputenc}
\usepackage[english,russian]{babel}
\usepackage{amssymb,amsfonts,amsmath,mathtext}
\usepackage{subfig}
\usepackage[all]{xy} % xy package for diagrams
\usepackage{array}
\usepackage{multicol}% many columns in slide
\usepackage{hyperref}% urls
\usepackage{hhline}%tables

% Your figures are here:
\graphicspath{ {../fig/} }

%----------------------------------------------------------------------------------------------------------
\title[\hbox to 56mm{Итеративное улучшение}]{Итеративное улучшение тематической \\ модели с обратной связью от пользователя}
\author[А.\,И. Горбулев]{Алексей Ильич Горбулев}
\institute{Московский физико-технический институт}
\date{\footnotesize
\par\smallskip\emph{Курс:} Моя первая научная статья/Группа Б05-021а
% Консультировался с В. А. Алексеевым по следующему поводу, так как схема такова: студент с Алексеевым, Алексеев с Воронцовым
\par\smallskip\emph{Эксперт:} д. ф.-м. н. К.\,В.~Воронцов
\par\smallskip\emph{Консультант:} В.\,А.~Алексеев
\par\bigskip\small 2023}
%----------------------------------------------------------------------------------------------------------
\begin{document}
%----------------------------------------------------------------------------------------------------------
\begin{frame}
\thispagestyle{empty}
\maketitle
\end{frame}
%-----------------------------------------------------------------------------------------------------
\begin{frame}{Итеративное улучшение модели}

{\footnotesize \textbf{Мотивация:} большая часть тем может вообще быть не связана с исследованием, который проводит пользователь}

{\footnotesize \textbf{Цель:} улучшить тематическую модель, сохранив ранее найденные релевантные темы, найти новые релевантные темы за счёт других}

{\footnotesize \textbf{Решение:} при обучении новой тематической модели:}
\begin{itemize}
        \item {\footnotesize при инициализации матрицы $\Phi$ зафиксировать столбцы, соответствующие релевантным темам}
        \item {\footnotesize использовать аддитивную регуляризацию тематической модели (ARTM):}
        \begin{itemize}
            \item {\footnotesize использовать регуляризатор декоррелирования, чтобы уменьшить число дублирования тем $$R(\Phi) = - \frac{\tau}{2} \sum \limits_{t \in T} \sum \limits_{s \in T \setminus t} \sum \limits_{w \in W} \varphi_{wt} \varphi_{ws}$$}
            \item {\footnotesize использовать регуляризаторы сглаживания и разреживания для улучшения интерпретируемости тем}
        \end{itemize}
    \end{itemize}


\end{frame}


%----------------------------------------------------------------------------------------------------------
\end{document} 